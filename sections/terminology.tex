\subsection{Terminology}
\begin{description}
	\item[Block access] 
		An access to a block of secondary storage. Since accessing secondary storage
		presents itself as the bottleneck in database latency, time complexity of
		database operations is often represented by the number of block accesses rather
		than the number of operations in RAM.
    \item[Data client]
    	A person or party collecting data for learning analytics purposes.
    \item[Data subject]
    	A person who contributes data to or is the subject of a learning analytics 
    	application. In the context of the survey tool, this is the person who
    	responds to a survey.
    \item[Learning management system]
	    A content management system, specifically designed for e-learning applications.
	    Examples for LMSs are Moodle and OLAT.
    \item[Learning record store]
    	A database system, possibly including analysis and visualisation capabilities, 
	    for storing data of interest to learning analytics applications.
	    In the context of this thesis, LRS refers to a system for storing and analysing xAPI statements in particular.
    	Examples for LRS systems are the TLA facts engine an HT2Labs's Learning Locker \ref{ht2labs-learninglocker}.
    \item[Survey item]
    	An organisational unit of a survey. In the context of this thesis,
    	a survey item may either be an entire questionnaire, a questionnaire's
    	dimension or a single question. 
    \item[Tool provider]
	    In the context of LTI, the term \textit{tool provider} is used to describe a
	    system which provides an external tool to an LMS, extending the LMS's capabilities.
\end{description}