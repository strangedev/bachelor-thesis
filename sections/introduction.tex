\section{Introduction}

\subsection{Previous Work}
    As part of the computational humanities seminar, conducted by \profhd in the 2017/18 winter semester, Hannes Leutloff
    and I were tasked to digitize the evaluation framework for learning analytics (EFLA) \cite{efla} and to
    develop an online platform where the survey could be taken.

    \begin{figure}
        \centering
        \includegraphics[width=\textwidth]{EFLA--greyscale}
        \caption{Template for the EFLA survey \cite{efla-website}}
        \label{fig:efla-template}
    \end{figure}

   The result is an online survey platform, where surveys similar to the EFLA can be created
   and hosted. While it is possible to create arbitrary survey items, some restrictions
   specific to the EFLA use-case apply (for a full list of features see Table \ref{table:v1-features}).
   The original version of the survey tool is written in Python 3 and JavaScript for the server
   and user interface respectively. To be able to re-use code, the choice of language did not
   change with the new version.
   

   \begin{figure}
       \begin{tabularx}{\textwidth}{|l|X|}
            \hline
            Feature & Description \\
            \hline \hline
            Editable surveys & Edit titles, question texts and colors.\\
            Export to CSV & Export all results of a questionnaire to as CSV file\\
            Challenges & Validate data subject's email addresses. 
            White- and Blacklist email addresses. Protect surveys by password.\\
            Internationalisation & Survey items may have multiple translations.\\
            User accounts & Users may sign up and host surveys.\\
            Template surveys & Choose from a fixed set of template surveys.\\
            Visualization & View survey results as a box plot.\\
            \hline
       \end{tabularx}
       \caption{Features of the previous version of the survey tool}
       \label{table:v1-features}
   \end{figure}

   \subsubsection{Data Model}

   \begin{figure}[H]
        \centering
        \includegraphics[width=\textwidth]{v1-model}
        \caption{Data model of the previous version}
        \label{fig:v1-data-model}
    \end{figure}

    \begin{wrapfigure}{o}{.5\textwidth}
        \centering
        \includegraphics[width=0.4\textwidth]{v1-stack}
        \caption{Architecture overview of the previous version}
        \label{fig:v1-stack}
    \end{wrapfigure}

    The data model for the previous version is closely coupled to the specific needs of the
    EFLA survey, where a single survey consists of two questionnaires, targeting
    learners and teachers respectively.
    Each questionnaire controls the submission right of data subjects by its own set of access 
    control modules, so that audience groups can be differentiated.
    Each questionnaire contains one or more question groups, which consists of
    one or more questions.

    \subsubsection{Architecture}
        The architecture for the previous version follows a simple client-server paradigm, where
        a web browser assumes the role of the client. The server software consists of an
        application server, responding to API requests, a database and an in-memory
        key-value store for storing session data. There are two different user interfaces,
        one for data clients and one for data subjects. The data subjects' user interface
        is the page, where the survey is filled out and submitted. This page is rendered
        on the server using a templating engine and sent to the browser as a static HTML document.
        The user interface for data clients presents the user with an editor, where survey content
        can be created and modified. This user interface is realized as a one-page
        JavaScript application and was developed by Hannes Leutloff. Deployment is handled through
        Docker, with the help of the docker-compose tool.

\subsection{Terminology}
\begin{description}
	\item[Block access] 
		An access to a block of secondary storage. Since accessing secondary storage
		presents itself as the bottleneck in database latency, time complexity of
		database operations is often represented by the number of block accesses rather
		than the number of operations in RAM.
    \item[data subject]
    \item[data client]
    \item[survey item] 
    \item[Tool provider]
	    In the context of LTI, the term \textit{tool provider} is used to describe a
	    system which provides an external tool to an LMS, extending the LMS's capabilities.
    \item[Learning record store]
    	A database system, possibly including analysis and visualisation capabilities, 
	    for storing data of interest to learning analytics applications.
	    In the context of this thesis, LRS refers to a system for storing and analysing xAPI statements in particular.
    	Examples for LRS systems are the TLA facts engine an HT2Labs's Learning Locker \ref{ht2labs-learninglocker}.
    \item[Learning management system]
	    A content management system, specifically designed for e-learning applications.
	    Examples for LMSs are Moodle and OLAT.
\end{description}

\subsection{The TLA Ecosystem}
    The current efforts of Prof. H. Drachsler's work group are targeted towards the implementation
    of a trusted learning analytics (TLA) infrastructure. This infrastructure will
    provide big data storage and analysis capabilities while complying with the European General
    Data Protection Regulation (GDPR). Data providers present a central part of this infrastructure.
    These data providers are tools used for collecting data and publishing it to
    the TLA facts store, using the xAPI statement format as common data representation (CDR).
    The aim here is not only to create the infrastructure for collecting and analyzing data,
    but also to build an ecosystem of external tools and data providers
    to interface with the TLA core's components.

\subsection{Motivation}
    By no stretch of the imagination are online survey tools a new invention.
    An online search yields dozens of already existing survey platforms, both
    commercial and free to use. Some of the biggest competitors in this business
    are compared in Table \ref{table:competitors}. For most use cases, one
    of the existing solutions will be sufficient. The survey tool described here
    aims at solving a unique challenge not solved by any competitor that the author
    is aware of. This challenge is integration with Moodle and the TLA
    infrastructure for use at the Goethe University in Frankfurt.
    Features needed for this integration are single sign-on, i.e.
    automated authentication of survey participants between Moodle and
    the survey tool, support for embedding via the LTI protocol, 
    as well as support for xAPI as a data exchange protocol.
    These requirements are discussed in greater detail in the Section
    ``Requirements Analysis''. While most survey providers provide single
    sign-on features, these capabilities are generally only provided to enterprise
    customers. Furthermore, only one of the competitors examined in this thesis
    supports CAS as a mechanism for this. None of the competitors
    support data export as xAPI statements out of the box and require either
    manual programming for every survey or data conversion after export. 
    Of the listed competitors, only one supports the LTI protocol for
    embedding the survey into Moodle and the only documentation of this 
    feature is an online video \cite{qualtrics-lti-video}.

    \subsubsection{Trusted Learning Analytics: Cloud Versus On-Premise}
        Another motivation for developing a new software instead of customizing
        one of the existing solutions is the amount of control this approach allows
        over the collected data. The competitors provide cloud-based software-as-a-service
        solutions, which do not allow for full control over the data collected on data subjects
        and their usage of the site. This is a serious disadvantage for applications
        in trusted learning analytics, as data clients cannot provide data subjects
        with autonomy over their data. One of the great challenges in learning analytics
        is fostering trust in the used data analysis methods and the ethical use
        of collected data. Ethical boundaries differ from person to person,
        and data subjects may not be comfortable with specific parts of their
        personal data being collected and analyzed. For these reasons, providing 
        transparency and allowing data subjects agency over the treatment of their personal data are
        cornerstones of the trusted learning analytics approach.
        With cloud-based providers, there are usually no guarantees about where
        the data is being stored and how long it is being stored for.
        Some providers, e.g. SoSci Survey, have already recogized this market niche 
        and advertise server hosting in Germany and encrypted data backups \cite{soscisurvey}.
        This does, however, not satisfy the requirements for transparency trusted
        learning analytics strives to achieve, as the infrastructure remains a
        black box. In addition, using on-premise software allows for greater agility,
        should requirements on data storage and handling change in the future.


    %\begin{landscape}
    {\renewcommand{\arraystretch}{1.5}
        \begin{table}
            \begin{tabularx}{1\textwidth}{|l|X|X|X|}
                \hline
                \diagbox{Feature}{Provider} & Google Forms & SurveyMonkey & QuestionPro \\
                \hline 
                SSO & For managed accounts in G-Suite via SAML and LDAP \cite{googleforms-sso}
                    & Supported, but further information is only provided to potential customers. \cite{surveymonkey-sso}
                    & Through SAML and HMAC-SHA1 \cite{questionpro-sso}
                    \\
                xAPI & Not supported, but has limited support for programmable event handlers. \cite{googleforms-webhooks}
                     & Not supported, but supports polling of survey results through an API. \cite{surveymonkey-api}
                     & Not supported, but has support for web hooks. \cite{questionpro-webhooks}
                     \\
                LTI & Unsupported 
                    & Unsupported 
                    & Unsupported
                    \\
                \hline
            \end{tabularx}
            \vskip 1em
            \begin{tabularx}{1\textwidth}{|l|X|X|}
                \hline
                \diagbox{Feature}{Provider} & qualtrics & survey gizmo \\
                \hline 
                SSO 
                    & Through CAS, LDAP or SAML \cite{qualtrics-sso}
                    & Through SAML \cite{surveygizmo-sso} \cite{surveygizmo-sso2}
                    \\
                xAPI 
                     & Not supported, but supports polling of survey results through an API. \cite{qualtrics-api}
                     & Not supported, but has limited support for web hooks. \cite{surveygizmo-webhooks}
                     \\
                LTI
                    & Alledged experimental support, no official documentation exists. 
                    & Unsupported
                    \\
                \hline
            \end{tabularx}
            \caption{Comparison of survey providers}
            \label{table:competitors}
        \end{table}
    }
    %\end{landscape}

\subsection{Research question}
    In the light of the need for an on-premise survey solution, which integrates with
    existing and emerging educational and learning analytics technologies, this
    thesis aims at answering the following research questions:

    \begin{itemize}
        \item[1)] How can the xAPI specification be applied to data collection in the context
            of an online survey tool? More specifically, how may the data be represented using
            the xAPI statement format and what technical difficulties arise when integrating the
            specification?
        \item[2)] Using the survey tool, how can data clients provide their survey items to others 
            and how can existing surveys be re-used?
        \item[3)] How can interoperability between existing educational technologies, more specifically
            learning management systems like Moodle, and the survey tool be achieved?
    \end{itemize}
